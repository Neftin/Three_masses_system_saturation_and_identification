%%%%%%%%%%%%%%%%%%%%%%%%%%%%%%%%%%%%%%%%%%%%%%%
%
% Template per Elaborato di Laurea
% DISI - Dipartimento di Ingegneria e Scienza dell’Informazione
%
% update 2015-09-10
%
% Per la generazione corretta del 
% pdflatex nome_file.tex
% bibtex nome_file.aux
% pdflatex nome_file.tex
% pdflatex nome_file.tex
%
%%%%%%%%%%%%%%%%%%%%%%%%%%%%%%%%%%%%%%%%%%%%%%%

% formato FRONTE RETRO
\documentclass[epsfig,a4paper,10pt,titlepage,twoside,openany]{book}

%\usepackage[top=2cm,bottom=2cm,inner=1.5cm,outer=1.5cm]{geometry}       %geometry
\renewcommand{\familydefault}{\sfdefault}
\usepackage[sfdefault, light]{roboto}

\usepackage[a-1b]{pdfx}

%\usepackage[italian]{babel}                %font/language
\usepackage[T1]{fontenc}
\usepackage[utf8]{inputenc}

\usepackage{amsmath}                       %maths
\usepackage{bm}
\usepackage{amssymb}					   %for numbersets

\usepackage{graphicx}
\graphicspath{{img/}}                      % folder delle immagini
\usepackage{epstopdf} 
\usepackage{float}
\usepackage{subfigure}

\usepackage{tabularx}
\usepackage{multirow}

%\usepackage{color}
\usepackage[dvipsnames,cmyk]{xcolor}            %colors


\usepackage{listings} 




%  EXOTIC STUFF %%%%%%%%%%%%%%%%%%%%%%%%%%%%%%%%%%%%%%%%%%%%%%%%%%
\newcommand*{\TakeFourierOrnament}[1]{{%                         %
		\fontencoding{U}\fontfamily{futs}\selectfont\char#1}}    %
\newcommand*{\danger}{\TakeFourierOrnament{66}}                  %          
                                                                 %
\usepackage{newunicodechar}

\newcommand\Warning{%
	\makebox[1.4em][c]{%
		\makebox[0pt][c]{\raisebox{.1em}{\small!}}%
		\makebox[0pt][c]{\color{red}\Large$\bigtriangleup$}}}

\newunicodechar{⚠}{\Warning}
%%%%%%%%%%%%%%%%%%%%%%%%%%%%%%%%%%%%%%%%%%%%%%%%%%%%%%%%%%%%%%%%%%

\newcommand{\norm}[1]{\left \lvert \left \lvert #1 \right \rvert \right  \rvert}

%rapid file input
\newcommand{\rs}[1]{\input{result/#1.txt}}

%create new scatola environment with enumeration
\newcounter{comment}[section]
\newenvironment{comment}[1][]{\refstepcounter{comment}\par\medskip
	\noindent \textbf{Comment~\thecomment. #1} }{\par}

\newcounter{box}
\newenvironment{scatola}[1]
{\refstepcounter{box}
\begin{center}
		\begin{tabular}{|p{\linewidth}|}
			\hline \textbf{Box~\thebox} #1  \\
		}
		{ 
			 \\ \hline
		\end{tabular} 
	\end{center}
 }

%%%%%%%%%%%%%%%%%%%%%%%%%%%%%%%%%%%%%%%%%%%%%%%%%%%%%%%% date

\date{today}


%%%%%%%%%%%%%%%%%%%%%%%%%%%%%%%%%%%%%%%%%%%%%%%%%%%%%%%%% math shortcuts

\newcommand{\tr}{^{{\bm \top}}}

\newenvironment{sistema}%
{\left\lbrace\begin{array}{@{}l@{}}}%
	{\end{array}\right.}

\newcommand{\veton}{\begin{bmatrix}} %start a vector
\newcommand{\vetoff}{\end{bmatrix}}   %end a vector
\newcommand{\maton}{\begin{bmatrix}}   %start a matrix
\newcommand{\matoff}{\end{bmatrix}}  %end a matrix 
%%%%%%%%%%%%%%%%%%%%%%%%%%%%%%%%%%%%%%%%%%%%%%%%%%%%%%%%% section titling

\usepackage[explicit]{titlesec} % per formato custom dei titoli dei capitoli

%%% titles colors

\definecolor{DarkBlue}{RGB}{0,0,102}
\definecolor{DarkColdGray}{RGB}{100,100,150}
\definecolor{LimeGreen}{cmyk}{.50,0,1,.20}

\newcommand{\colorChapter}{LimeGreen} % color of the Chapter title 
\newcommand{\colorSection}{DarkBlue}  % color of the Section title
\newcommand{\colorSubsection}{Gray}   %color of the subsection title

%%% titles format

    \titleformat{\chapter}
{\normalfont\Huge\bfseries}{}{1em}{\textcolor{\colorChapter}{\thechapter} \textcolor{\colorSection}{|} \filcenter \textcolor{\colorChapter}{#1}}

    \titleformat{\section}
{\normalfont\large\bfseries}{\textcolor{\colorSection}{#1} }{1em}{\hfill \textcolor{\colorSection}{\thesection}}

    \titleformat{\subsection}[wrap]
{\normalfont\large\bfseries}{\textcolor{\colorSubsection}{#1}}{1em}{\\ \small \textcolor{\colorSubsection}{\thesubsection}}

%%% titles spacing {left}{above}{below}

\titlespacing*{\chapter}{0pt}{1em}{1em}
\titlespacing*{\section}{0pt}{1em}{1em}
\titlespacing*{\subsection}{0pt}{1em}{1em}

%%% 

 %my preamble

\usepackage{plain}
\usepackage{setspace}
%\usepackage[paperheight=29.7cm,paperwidth=21cm,outer=1.5cm,inner=2.5cm,top=2cm,bottom=2cm]{geometry} % per definizione layout (DISI standard)


\singlespacing

\usepackage[italian]{babel}

%%%  titolo ed altre frasi ricorrenti
\newcommand{\mytitle}{Sintesi e validazione sperimentale di tecniche di controllo anti-windup basate su linear matrix inequalities }
\newcommand{\treM}{sistema masse-molle } % compatibile con articolo 'il'

\begin{document}


  % nessuna numerazione
  \pagenumbering{gobble} 
  \begin{center}
\pagestyle{plain}

\thispagestyle{empty}

\begin{figure}
	\centering
	\includegraphics[width=\linewidth]{img/logo_unitn_black_center}
\end{figure}


  \vspace{2 cm} 

  \LARGE{Dipartimento di Ingegneria Industriale\\}

  \vspace{1 cm} 
  \Large{Corso di Laurea in Ingegneria Meccatronica}

  \vspace{2 cm} 
%  \Large\textsc{Elaborato finale\\} 
%  \vspace{1 cm} 
  \Huge\textsc{\mytitle}
%  \Large{\it{Sottotitolo (alcune volte lungo - opzionale)}}


  \vspace{2 cm} 
  \begin{tabular*}{\textwidth}{ c @{\extracolsep{\fill}} c }
  \Large{Supervisore} & \Large{Laureando}\\
  \Large{Luca Zaccarian}& \Large{Giammarco Valenti}\\
  \end{tabular*}

  \vspace{1 cm} 

  \Large{Anno accademico 2016/2017}
  
\end{center}



  \clearpage
 
%% Sezione Ringraziamenti opzionale

  %\input{ringraziamenti}
  \clearpage
  \pagestyle{plain} % nessuna intestazione e pie pagina con numero al centro

  
  % inizio numerazione pagine in numeri arabi (corpo tesi)
  \mainmatter

    % indice
    \tableofcontents
    \clearpage
    
    
          
    % gruppo per definizone di successione capitoli senza interruzione di pagina
    \begingroup
      % nessuna interruzione di pagina tra capitoli
      % ridefinizione dei comandi di clear page
      \renewcommand{\cleardoublepage}{} 
      \renewcommand{\clearpage}{} 
      % redefinizione del formato del titolo del capitolo
      % da formato
      %   Capitolo X
      %   Titolo capitolo 
      % a formato
      %   X   Titolo capitolo
        

      
      % sommario
      %\input{sommario}  
      
      %%%%%%%%%%%%%%%%%%%%%%%%%%%%%%%%
      % lista dei capitoli
      %
      % \input oppure \include
      %
      %\input{Introduzione}
      \chapter{Il benchmark}
Il problema della saturazione verrà affrontato su un sistema meccanico reale. Il sistema nasce a scopi didattici per provare architetture di controllo su un sistema lineare. Una foto dell'apparato sperimentale è mostrata in Figura \ref{fig:fotosetup}.
\begin{figure}
	\centering
	\includegraphics[height=0.5\linewidth]{img/foto_setup}
	\caption{Foto dell'apparato sperimentale (\treM)}
	\label{fig:fotosetup}
\end{figure}
\section{Descrizione dell'apparato sperimentale}
Il \treM si compone di tre carrelli posizionati su tre binari. I binari sono allineati e perciò i carrelli sono vincolati a muoversi solo lungo questo asse. Ogni carrello prevedere la possibilità di essere caricato cn un numero da $0$ a $4$ masse da circa $500 \ g$.
%TODO vedi se aggiungere foto di carrello singolo e pinion rack




      
% from LUYA constraint to LMI

\chapter{LMI per uomini veri}

Partiamo dalla definizione generale del sistema con solamente lo spazio di stato senza inut ma solo con un input d che può essere un disturbo o un riferimento di errore. (ad esempio se chiudiamo il loop se abbiamo un segnale riferimento $w(t)$ d può ricoprire il ruolo di $d(t) = w(t) - y(t)$ dove y è il feedback del valore di output da portare a riferimento). Il sistema è definitio in \eqref{eq:sys_gen}.
\begin{equation}
\label{eq:sys_gen}
	\begin{sistema}
	\dot{x} = \bar{A}x + \bar{B}_d d\\
	z       = \bar{C}x + \bar{D}_d d
	\end{sistema}
\end{equation}
Definiamo la funzione quadratica di Luyapunov  $V(x) = x\tr P x$. la condizione di negatività sulla derivata prima nel tempo garantisce la Global Exponential Stability (GES) del sistema ($\dot{V}(x) < 0$). La condizione si può ``estendere'' per comprendere l'andamento di $z$ rispetto a $w$, la condizione estesa è mostrata in \eqref{eq:imbuto_sys_gen}.
\begin{equation}
\label{eq:imbuto_sys_gen}
	\dot{V} + \dfrac{1}{\gamma^2}z\tr z - w\tr w < 0
\end{equation}
%TODO capire perchè si aggiungono così secchi z e w l_2
La condizione \eqref{eq:imbuto_sys_gen} implica un limite superiore della norma di $z$. L'implicazione si ottiene integrando l'equazione \eqref{eq:imbuto_sys_gen} da $0$ a $\infty$, come indicato dalle equazioni: \eqref{eq:imbuto_sys_gen_int} e \eqref{eq:L2gain_sys_gen}. Il sistema è GES perciò $V(\infty) = 0$. 
\begin{equation}
\label{eq:imbuto_sys_gen_int}
	\int_{0}^{\infty}\left (V(x) - V(0) + \dfrac{1}{\gamma^2}z\tr z - w\tr w    \right )dt < 0
\end{equation}
\begin{equation}
\label{eq:L2gain_sys_gen}
	\norm{z}_2^2 \leq \gamma^2\norm{w}^2_2 + \gamma^2 V(0)
\end{equation}
%TODO schema impianto solo d e z con H centrale di stati x, poi aggiungiamo la nonlinearità e poi lo antiwindappo
La condizione \eqref{eq:imbuto_sys_gen} può essere scritta esplicitando la funzione quadratica di Luyapunov:
\begin{equation}
\label{eq:imbuto_sys_gen_luya}
	2x\tr P\left  (\bar{A}x + \bar{B}_dd\right ) +  \dfrac{1}{\gamma^2}z\tr z - w\tr w < 0
\end{equation}
Ora raccogliendo il vettore $ \veton & x & d & \vetoff \tr $ si ottiene:
\begin{equation}
	2  \veton x \\ d \vetoff \maton P\bar{A} & P \bar{B}_d \\ 0 & -I/2 \matoff  \veton x \\ d \vetoff 
	+ \dfrac{1}{\gamma^2} \left ( \veton \bar{C} & \bar{D}_d \vetoff \veton x \\ d \vetoff \right ) \tr \left  ( \veton \bar{C} & \bar{D}_d \vetoff \veton x \\ d \vetoff \right ) < 0 
\end{equation}
\begin{equation}
	\maton \bar{A}P + P\bar{A} & P \bar{B}_d \\ 0 & -I/2 \matoff + \dfrac{1}{\gamma^2}\veton \bar{C} \tr \\ \bar{D}_d \tr \vetoff \veton \bar{C}  & \bar{D}_d \vetoff
\end{equation}
\begin{equation}
	He\maton \bar{A}P + P\bar{A} & P \bar{B}_d & 0 \\ 0 & -I & 0 \\ \bar{C} & \bar{D}_d & -\gamma^2 I  \matoff < 0
\end{equation}
%TODO definisci operatore He



      %\input{Sintesi}
      %\input{anti_wind_up}
      
      
    \endgroup


    % bibliografia in formato bibtex
    %
    % aggiunta del capitolo nell'indice
    \addcontentsline{toc}{chapter}{Bibliografia}
    % stile con ordinamento alfabetico in funzione degli autori
    \bibliographystyle{plain}
    \bibliography{biblio}
    
   
        
    % sezione Allegati - opzionale
    %\appendix
    %\input{allegati}

\end{document}
