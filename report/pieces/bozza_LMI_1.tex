
% from LUYA constraint to LMI

\chapter{LMI per uomini veri}

Partiamo dalla definizione generale del sistema con solamente lo spazio di stato senza inut ma solo con un input d che può essere un disturbo o un riferimento di errore. (ad esempio se chiudiamo il loop se abbiamo un segnale riferimento $w(t)$ d può ricoprire il ruolo di $d(t) = w(t) - y(t)$ dove y è il feedback del valore di output da portare a riferimento). Il sistema è definitio in \eqref{eq:sys_gen}.
\begin{equation}
\label{eq:sys_gen}
	\begin{sistema}
	\dot{x} = \bar{A}x + \bar{B}_d d\\
	z       = \bar{C}x + \bar{D}_d d
	\end{sistema}
\end{equation}
Definiamo la funzione quadratica di Luyapunov  $V(x) = x\tr P x$. la condizione di negatività sulla derivata prima nel tempo garantisce la Global Exponential Stability (GES) del sistema ($\dot{V}(x) < 0$). La condizione si può ``estendere'' per comprendere l'andamento di $z$ rispetto a $w$, la condizione estesa è mostrata in \eqref{eq:imbuto_sys_gen}.
\begin{equation}
\label{eq:imbuto_sys_gen}
	\dot{V} + \dfrac{1}{\gamma^2}z\tr z - w\tr w < 0
\end{equation}
%TODO capire perchè si aggiungono così secchi z e w l_2
La condizione \eqref{eq:imbuto_sys_gen} implica un limite superiore della norma di $z$. L'implicazione si ottiene integrando l'equazione \eqref{eq:imbuto_sys_gen} da $0$ a $\infty$, come indicato dalle equazioni: \eqref{eq:imbuto_sys_gen_int} e \eqref{eq:L2gain_sys_gen}. Il sistema è GES perciò $V(\infty) = 0$. 
\begin{equation}
\label{eq:imbuto_sys_gen_int}
	\int_{0}^{\infty}\left (V(x) - V(0) + \dfrac{1}{\gamma^2}z\tr z - w\tr w    \right )dt < 0
\end{equation}
\begin{equation}
\label{eq:L2gain_sys_gen}
	\norm{z}_2^2 \leq \gamma^2\norm{w}^2_2 + \gamma^2 V(0)
\end{equation}
%TODO schema impianto solo d e z con H centrale di stati x, poi aggiungiamo la nonlinearità e poi lo antiwindappo
La condizione \eqref{eq:imbuto_sys_gen} può essere scritta esplicitando la funzione quadratica di Luyapunov:
\begin{equation}
\label{eq:imbuto_sys_gen_luya}
	2x\tr P\left  (\bar{A}x + \bar{B}_dd\right ) +  \dfrac{1}{\gamma^2}z\tr z - w\tr w < 0
\end{equation}
Ora raccogliendo il vettore $ \veton & x & d & \vetoff \tr $ si ottiene:
\begin{equation}
	2  \veton x \\ d \vetoff \maton P\bar{A} & P \bar{B}_d \\ 0 & -I/2 \matoff  \veton x \\ d \vetoff 
	+ \dfrac{1}{\gamma^2} \left ( \veton \bar{C} & \bar{D}_d \vetoff \veton x \\ d \vetoff \right ) \tr \left  ( \veton \bar{C} & \bar{D}_d \vetoff \veton x \\ d \vetoff \right ) < 0 
\end{equation}
\begin{equation}
	\maton \bar{A}P + P\bar{A} & P \bar{B}_d \\ 0 & -I/2 \matoff + \dfrac{1}{\gamma^2}\veton \bar{C} \tr \\ \bar{D}_d \tr \vetoff \veton \bar{C}  & \bar{D}_d \vetoff
\end{equation}
\begin{equation}
	He\maton \bar{A}P + P\bar{A} & P \bar{B}_d & 0 \\ 0 & -I & 0 \\ \bar{C} & \bar{D}_d & -\gamma^2 I  \matoff < 0
\end{equation}
%TODO definisci operatore He


